\documentclass[natbib]{svjour3}

% UTF8 support
\usepackage[utf8x]{inputenc}

\usepackage[english]{babel}


\usepackage{graphicx}
\graphicspath{{figs/}}

\usepackage{amsmath}

\usepackage{tikz}
\usetikzlibrary{shapes}

\usepackage{hyperref}
\usepackage{subcaption}

\usepackage[draft, footnote, margin=false]{fixme}

\newcommand{\ie}{{\textit{i.e.\ }}}
\newcommand{\cf}{{\textit{cf\ }}}
\newcommand{\eg}{{\textit{e.g.\ }}}
\newcommand{\al}{{\textit{et al.\ }}}

\usepackage{xspace}
\newcommand{\A}{A\xspace}
\newcommand{\B}{B\xspace}
\newcommand{\C}{C\xspace}


\newcommand{\M}[3]{{\mathcal{M}(#1, #2, #3)}}
\newcommand{\model}[3]{{$\mathcal{M}(#1, #2, #3)$}}
% generic mutual model (gmodel) 
\newcommand{\gmodel}[2]{{$\mathcal{M}(#1, #2)$}}
\newcommand{\Gmodel}[2]{{\mathcal{M}(#1, #2)}}
\newcommand{\refmodel}[2]{{$\mathcal{M}(#1, #2)$}}
\newcommand{\Model}[3]{{$\mathcal{M}^{\circ}(#1, #2, #3)$}}
\newcommand{\gModel}[2]{{$\mathcal{M}^{\circ}(#1, #2)$}}
\newcommand{\Mdeg}[3]{{\mathcal{M}^{\circ}(#1, #2, #3)}}
\newcommand{\gMdeg}[2]{{\mathcal{M}^{\circ}(#1, #2)}}

\newcommand{\groundingcriterion}{{$\mathcal{M}^{\circ}_{min}$}}
\newcommand{\inigrounding}{{$\mathcal{M}^{\circ}_{t_0}$}}

\newcommand{\concept}[1]{{\small \texttt{#1}}}
\newcommand{\stmt}[1]{{\footnotesize \tt $\langle$ #1\relax$\rangle$}}

\usepackage{mathtools}
\DeclarePairedDelimiter\abs{\lvert}{\rvert}%
\newcommand*\mean[1]{\overline{#1}}

\title{The Symmetry of Partner Modelling}

\author{
    Pierre Dillenbourg
    \and
    Séverin Lemaignan
    \and
    Mirweis Sangin
    \and
    Nicolas Nova
    \and
    Gaëlle Molinari
}
\institute{
    Pierre Dillenbourg, Séverin Lemaignan \at Computer-Human Interaction in Learning and Instruction \\
    École Polytechnique Fédérale de Lausanne (EPFL) \\ CH-1015 Lausanne, Switzerland \\
    \email{firstname.lastname@epfl.ch}
    \and
    Mirweis Sangin \at Use-able consulting \\ CH-1004 Lausanne, Switzerland\\\email{mirweis@gmail.com}
    \and
    Nicolas Nova \at Haute École Arts \& Design de Genève \\ CH-1201 Genève, Switzerland\\\email{nicolas.nova@hesge.ch}
    \and
    Gaëlle Molinari \at Distance Learning University Switzerland (Unidistance),
    CH-3960 Sierre, Switzerland\\\email{gaelle.molinari@unidistance.ch}
}
\begin{document}
\maketitle

\begin{abstract}
Collaborative learning has often been associated to the construction of  a
shared   understanding of the situation at hand. The psycholinguistics
mechanisms at work while establishing common grounds are the object of
scientific controversy. We postulate that collaborative tasks require some
level of mutual modelling, \ie that each partner needs some model of what
the other partners know/want/intend at a given time.  We use the term ``some
model'' to stress the fact that this model is not necessarily detailed or
complete, but that we acquire some representations of the persons we
interact with. The question we address is: Does the quality of the partner
model depend upon the modeler's ability to represent his or her partner?
Upon the modelee's ability to make his state clear to the modeler? Or rather upon
the quality of their interactions? We address this question by comparing
the respective accuracies of the models built by different team
members. We therefore report on 5 experiments on collaborative problem solving
or collaborative learning that vary in terms of tasks (how important it is
to build an accurate model) and settings (how difficult it is to build an
accurate model). In 4 studies, the accuracy of the model that \A built about
\B was correlated with the accuracy of the model that \B built about \A,  which
seems to imply that the quality of interactions matters more than individual
abilities when building mutual models.  However, these finding do not
rule out the fact that individual abilities also participate in the quality of
modelling process.
\end{abstract}

\subsection*{Corresponding author}
Séverin Lemaignan \texttt{severin.lemaignan@plymouth.ac.uk}
\end{document}

