\documentclass[natbib]{svjour3}

% UTF8 support
\usepackage[utf8x]{inputenc}

\usepackage[english]{babel}


\usepackage{graphicx}
\graphicspath{{figs/}}

\usepackage{amsmath}

\usepackage{tikz}
\usetikzlibrary{shapes}

\usepackage{hyperref}
\usepackage{subcaption}

\usepackage[draft, footnote, margin=false]{fixme}

\newcommand{\ie}{{\textit{i.e.\ }}}
\newcommand{\cf}{{\textit{cf\ }}}
\newcommand{\eg}{{\textit{e.g.\ }}}
\newcommand{\al}{{\textit{et al.\ }}}

\newcommand{\M}[3]{{\mathcal{M}(#1, #2, #3)}}
\newcommand{\model}[3]{{$\mathcal{M}(#1, #2, #3)$}}
% generic mutual model (gmodel) 
\newcommand{\gmodel}[2]{{$\mathcal{M}(#1, #2)$}}
\newcommand{\Gmodel}[2]{{\mathcal{M}(#1, #2)}}
\newcommand{\refmodel}[2]{{$\mathcal{M}(#1, #2)$}}
\newcommand{\Model}[3]{{$\mathcal{M}^{\circ}(#1, #2, #3)$}}
\newcommand{\gModel}[2]{{$\mathcal{M}^{\circ}(#1, #2)$}}
\newcommand{\Mdeg}[3]{{\mathcal{M}^{\circ}(#1, #2, #3)}}
\newcommand{\gMdeg}[2]{{\mathcal{M}^{\circ}(#1, #2)}}

\newcommand{\groundingcriterion}{{$\mathcal{M}^{\circ}_{min}$}}
\newcommand{\inigrounding}{{$\mathcal{M}^{\circ}_{t_0}$}}

\newcommand{\concept}[1]{{\small \texttt{#1}}}
\newcommand{\stmt}[1]{{\footnotesize \tt $\langle$ #1\relax$\rangle$}}

\usepackage{mathtools}
\DeclarePairedDelimiter\abs{\lvert}{\rvert}%
\newcommand*\mean[1]{\overline{#1}}

\title{Mutual Modelling}

\author{
    Pierre Dillenbourg
    \and
    Séverin Lemaignan
    \and
    Mirweis Sangin
    \and
    Nicolas Nova
    \and
    Gaëlle Molinari
}
\institute{
    Pierre Dillenbourg, Séverin Lemaignan \at Computer-Human Interaction in Learning and Instruction \\
    École Polytechnique Fédérale de Lausanne (EPFL) \\ CH-1015 Lausanne, Switzerland \\
    \email{firstname.lastname@epfl.ch}
    \and
    Mirweis Sangin \at Use-able consulting \\ CH-1004 Lausanne, Switzerland\\\email{mirweis@gmail.com}
    \and
    Nicolas Nova \at Haute École Arts \& Design de Genève \\ CH-1201 Genève, Switzerland\\\email{nicolas.nova@hesge.ch}
    \and
    Gaëlle Molinari \at Distance Learning University Switzerland (Unidistance),
    CH-3960 Sierre, Switzerland\\\email{gaelle.molinari@unidistance.ch}
}
\begin{document}
\maketitle

\begin{abstract}
    Collaborative learning has something to do with building a shared
    understanding of the situation at hand. Yet, the psycholinguistics mechanisms
    at work while establishing common grounds are complex and certainly involve
    modelling what the partner knows/wants/intends at a given time.

    Building on a set of five independent studies that where conducted over
    several years, we propose in this article a simple model of the mechanisms
    of mutual modelling. It integrates several concepts related to the classical
    idea of grounding, and we apply it as an operational tool to
    analyse the interplay of learning and mutual modelling in the five studies.

    We come up with mixed experimental results that evidence the dual nature of
    mutual modelling: a individual meta-cognitive aptitude that co-exists with
    modelling elicited and fostered by social interactions.

\end{abstract}

\subsection*{Corresponding author}
Séverin Lemaignan \texttt{severin.lemaignan@epfl.ch}
\end{document}

