\documentclass{article}
\usepackage[utf8]{inputenc}


\begin{document}

Dear Editor, Dear Reviewers,

\vspace{2em}

We would like to thank both the reviewers and the editor for going through our
manuscript and providing in-depth reviews of our submission.

The main points that were raised relate to:
\begin{itemize}
    \item the need for a better presentation of the concepts that we build upon,
    \item the need for a deeper synthetic discussion (also accounting for the
        conflicting results) that would clarify the theoretical
        picture resulting from our work.
\end{itemize}

In line with these comments and the other suggestions made by the reviewers, the
manuscript underwent several significant changes that we summarize hereafter.

\begin{itemize}

    \item The title and the abstract have been changed to clarify the
        claims of the paper, which, as the editor pointed out, were not clear
        for some reviewers. In a nutshell, we do not know what is going on in the
        brain when building a shared solution (we leave this scientific debate
        to psycholinguists), but we observe how they vary (or not) across
        situations (across the 5 experiments) and this enables some reflexions
        that we put in the context of collaborative learning.

    \item \emph{Section 1 -- Introduction} We did minor edits except for the last
        paragraph that has been completely rewritten, as it was somewhat misleading
        regarding to the goals of the paper.  We think the point is now made
        clearly that the issue is not whether it is alway necessary to build a
        shared understanding, but that CSCL folds design tasks specifically for
        requiring shared understanding.

    \item \emph{Section 2} has been split into partner modelling and mutual modelling (now
        section 3) to emphasize the fact that the question is the mutuality. The
        main research question is then clearly stated so as to address the
        perceived lack of clarity raised by some reviewers.

    \item \emph{Section 'Notations'} We clarified the status of the proposed formalism, which some
        reviewers found useful and others criticized. We are not
        proposing here a computational theory of mutual modelling. This notation is
        only useful because the text become quickly unreadable as one talks about
        the model that A build about the model that B build about A, while
        M(A,B,M(B,A,X)) is less ambiguous. We mainly see it as a language for discussing
        experiments and hypotheses in a simple way. Its function is internal to
        this paper.

    \item As requested by the reviewer 2, we have also clarified how we estimate the
        grounding criterion (p.6), and clarified as well that we have not attempted to
        quantify the cost of grounding in the presented studies (p.8).

    \item \emph{Section 'Studies'} For all the 5 studies, we have re-written all
        the \emph{discussion} sub-sections in a
        way that they naturally align with the research questions, leading to a
        clearer overall message.

    \item As requested by the reviewers, we have entirely rewritten  the
        discussion: our results are not clearcut (and, in that regard, we argue
        that as a community, we have to be careful not to publish only
        papers with significant results -- statisticians have proved that some
        meta-analyses are biased by the fact that papers with significant results
        get more easily published). However, the fact that none of our three
        main hypotheses (the quality of mutual modelling depends on the modeller's
        skills \emph{vs} it depends on the modellee's capability of making herself easy
        to model \emph{vs} it depends on the overall quality of the interaction)
        emerges as a clearly dominant factor is the key result of our work: we indeed show that
        individual skills only explain a part of the observed variance in
        collaborative tasks performance, and that the quality of
        social interactions does have a significant impact as well.

\end{itemize}

We do hope that the sum of the changes brought to the article clarifies and
strengthens our contribution, and we remain of course dedicated to account for further
suggestions if necessary.

\end{document}
