\documentclass{article}
\usepackage[utf8]{inputenc}


\begin{document}

Dear Editor, Dear Reviewers,

\vspace{2em}

We would like to thank both the reviewers and the editor for going through our
manuscript. As a follow-up of your comments, the manuscript underwent
several significant changes, that we summarize hereafter.

\begin{itemize}

    \item The title and the abstract have been completely changed to clarify the
        claims of the paper, which, as the editor pointed out, were not clear
        for some reviewers. In a nutshell, we don’t know what’s going on in he
        brain when building a shared solution, we leave this scientific debate
        to psycholinguists, but we observe how they vary (or not) across
        situations (across the 5 experiments)  and this enables some inferences.

    \item   Section 1, Introduction. We did minor edits except for the last
        paragraph has been completely rewritten it  that was somehow misleading
        regarding to the goals of the paper.  I think the point is now made
        clearly that the issue is not whether it is alway necessary to build a
        shared understanding, but that CSCL folds design tasks specifically for
        requiring shared understanding.

    \item We clarified the notion of grounding criterion.

    \item Section 2 has been split into partner modeling and mutual modeling (now
        section 3), to emphasized the fact that the question is the mutuality. The
        main research question is then clearly stated to address some of the
        reviewers cirtciisms

    \item Section 4. We clarified the status of the proposed formalism, which some
        reviewers find convenient and others criticize. Clearly, we are not
        proposing here a computational theory of mutual modeling. This notation is
        only useful because the text become quickly unreadable as one talks about
        the model that A build about the model that B build about A, while
        M(A,B,M(B,A,X)) is clear. So, it’s just a language for discussing
        experiments and hypotheses in a simple way. It’s function is internal to
        this paper

    \item For the 5 studies, we have re-written all the ‘discussion’ sections in a
        way that they cumulate naturally in alignment with the research questions

    \item As requested by the reviewers, we have completely rewritten  the
        conclusions to make them more explicit.  We cannot of course not pretend
        that our results are clearcut. We have to be careful not to publish only
        papers with significant results (statisticians have proved that some
        meta-analyses are biased by the fact that papers with significant results
        get more easily published). However, despite the fact that none of the 3
        hypothesis is declared as winner, the key results is that we prove that
        individual skills only explain a part of the variance, that the quality of
        social interactions matter significantly. This is why Gerry asked me (a
        few years ago I must say) to present these results that I mentionned in a
        ICLS panel.

\end{itemize}

\end{document}
